\documentclass[11pt]{beamer}

\usetheme{Madrid}
%\usepackage{palatino} % ← Commenté temporairement

\setbeamertemplate{footline}{
  \leavevmode%
  \hbox{%
    \begin{beamercolorbox}[wd=0.4\paperwidth,ht=2.5ex,dp=1ex,left]{author in head/foot}%
      \usebeamerfont{author in head/foot}\hspace{1em}\insertshortauthor
    \end{beamercolorbox}%
    \begin{beamercolorbox}[wd=0.30\paperwidth,ht=2.5ex,dp=1ex,center]{title in head/foot}%
      \usebeamerfont{title in head/foot}\insertshorttitle
    \end{beamercolorbox}%
    \begin{beamercolorbox}[wd=0.3\paperwidth,ht=2.5ex,dp=1ex,right]{date in head/foot}%
      \usebeamerfont{date in head/foot}\insertshortdate{} \hspace{1em}
      \insertframenumber{} / \inserttotalframenumber\hspace{1em}
    \end{beamercolorbox}%
  }%
  \vskip0pt%
}

\setbeamertemplate{navigation symbols}{}
\setbeamercovered{transparent}

\title[Présentation]{Avancement SAE}
\subtitle{S2.01 Conception et implémentation d'une base de données}

\author[Ibrahim BENKHERFELLAH \and Axel COULET]{Ibrahim BENKHERFELLAH \and Axel COULET}
\institute[USPN]{Université Sorbonne Paris Nord \\}
\date[\today]{\today}

\begin{document}

\maketitle

\begin{frame}
  \frametitle{Table des matières}
  \tableofcontents
\end{frame}

\section{Introduction}
\begin{frame}
  \frametitle{Objectif du projet}
  \begin{itemize}
    \item<1-> Comprendre et modéliser le commerce des technologies à faible émission de carbone.
    \item<2-> Concevoir une base de données normalisée à partir de deux jeux de données CSV.
    \item<3-> Interroger et visualiser les données pour produire des analyses pertinentes.
  \end{itemize}
\end{frame}

\section{Exploration et compréhension des données}
\begin{frame}
  \frametitle{Étude des sources de données}
  \begin{itemize}
    \item Fichiers CSV étudiés :
      \begin{itemize}
        \item<1-> \texttt{\href{https://climatedata.imf.org/datasets/1d33174e9e46429d9e570d539556f66a/explore}{Trade\_in\_Low\_Carbon\_Technology\_Products.csv}}
        \item \texttt{\href{https://climatedata.imf.org/datasets/975bc577fe7342c2a3651e8841959c47_0/explore}{Bilateral\_Trade\_in\_Low\_Carbon\_Technology\_Products.csv}}
      \end{itemize}
    \item<2-> Exploration initiale avec Python (pandas) : \\
          \texttt{.head()}, \texttt{.info()}, etc.
    \item<3-> Identification de colonnes clés : \texttt{CTS\_Code}, \texttt{Indicator}, \texttt{Trade\_Flow}, etc.
    \item<4-> Difficultés rencontrées : [à compléter]
  \end{itemize}
\end{frame}

\begin{frame}
  \frametitle{Sources et structuration des fichiers}
  \begin{itemize}
    \item<1-> Données structurées en colonnes d'années : F1994 à F2023.
    \item<2-> Présence de colonnes de description redondantes (\texttt{CTS\_Code}, \texttt{CTS\_Name}, \texttt{CTS\_Full\_Descriptor}).
    \item<3-> Première étape : lecture manuelle pour repérage des redondances et structures tabulaires.
  \end{itemize}
\end{frame}

\begin{frame}
  \frametitle{Préparation et normalisation des données}
  \begin{itemize}
    \item<1-> Conversion des colonnes FXXXX en format court (année, valeur).
    \item<2-> \text{Normalisation des années (F1994 à F2023)} :
    \begin{itemize}
      \item Chaque colonne FXXXX devient une ligne avec une valeur d'année.
      \item Transformation effectuée via \texttt{UNION ALL} dans SQL pour correspondre à la logique de \texttt{stack()} en Python.
    \end{itemize}
    \item<3-> Problèmes rencontrés :
    \begin{itemize}
      \item Ambiguïté sur le pays origine.
      \item Duplication de lignes si filtrage incorrect.
    \end{itemize}
  \end{itemize}
\end{frame}

\begin{frame}
  \frametitle{Données structurées et décisions clés}
  \begin{itemize}
    \item<1-> Création d'une table \texttt{echanger\_avec} pour les données bilatérales :
    \begin{itemize}
      \item Relation réflexive entre deux pays (deux clés étrangères vers \texttt{country}).
      \item Attribution de \texttt{pays\_origine} selon le sens du flux.
    \end{itemize}
    \item<2-> Table \texttt{donnee\_pays} pour les données nationales.
    \item<3-> Lien systématique aux dimensions : pays, indicateur, catégorie, année.
  \end{itemize}
\end{frame}

\section{Modélisation de la base de données}
\begin{frame}
  \frametitle{Modèle conceptuel (EA)}
  \begin{itemize}
    \item<1-> Type Entités : \texttt{Country}, \texttt{Indicator}, \texttt{CTS}, \texttt{Trade\_Flow}, \texttt{Year}
    \item<2-> Type Associations : 
      \begin{itemize}
        \item \texttt{Échanger\_Avec} (réflexive sur \texttt{Country})
        \item \texttt{Donnée\_Pays} (relation simple avec agrégation)
      \end{itemize}
    \item<3-> Justification des cardinalités et des relations
    \item<4-> Illustration du schéma : 
  \end{itemize}
\end{frame}

\begin{frame}
  \frametitle{Structure relationnelle finale}
  \begin{itemize}
    \item Schéma relationnel résultant de la modélisation EA :
    \begin{itemize}
      \item \texttt{country(idCountry, Country, ISO2, ISO3)}
      \item \texttt{cts(idCTS, CTS\_Code, CTS\_Name, CTS\_Full\_Descriptor)}
      \item \texttt{trade\_flow(idTrade, Trade\_Flow, Scale)}
      \item \texttt{indicator(idIndicator, Indicator\_, Source, Units)}
      \item \texttt{year(id\_Year, Year)}
      \item \texttt{echanger\_avec(id\_country\_1, id\_country\_2, pays\_origine, id\_indicator, id\_cts, id\_trade, id\_year, trade\_value)}
      \item \texttt{donnee\_pays(id\_country, id\_indicator, id\_cts, id\_trade, id\_year, trade\_value)}
    \end{itemize}
  \end{itemize}
\end{frame}

\begin{frame}
  \frametitle{Forme normale : 1NF}
  \begin{itemize}
    \item Toutes les valeurs sont atomiques : aucun champ multivalué ou composé.
    \item Transformation des colonnes F1994 à F2023 en une colonne \texttt{id\_year} via pivot SQL.
    \item Tables relationnelles sans redondance horizontale.
    \item \textbf{Conclusion} : le schéma respecte la \textbf{première forme normale (1NF)}.
  \end{itemize}
\end{frame}

\begin{frame}
  \frametitle{Forme normale : 2NF}
  \begin{itemize}
    \item Le schéma est en 1NF.
    \item Aucune dépendance fonctionnelle partielle dans les tables à clés composites.
    \item Exemple : les descripteurs CTS ont été extraits dans une table distincte, reliée par \texttt{CTS\_Code}.
    \item \textbf{Conclusion} : toutes les dépendances fonctionnelles concernent la clé entière → \textbf{2NF validée}.
  \end{itemize}
\end{frame}

\begin{frame}
  \frametitle{Forme normale : 3NF}
  \begin{itemize}
    \item Le schéma est en 2NF.
    \item Suppression des dépendances transitives.
    \item Exemple : l’unité d’un indicateur dépend directement de \texttt{indicator}, pas d’un pays ou d’un code CTS.
    \item \textbf{Conclusion} : toutes les colonnes non-clés dépendent uniquement de la clé primaire → \textbf{schéma en 3NF}.
  \end{itemize}
\end{frame}

\begin{frame}
  \frametitle{Forme normale : BCNF}
  \begin{itemize}
    \item Toutes les dépendances fonctionnelles ont un antécédent qui est une super-clé.
    \item Exemple : dans \texttt{echanger\_avec}, seule la combinaison complète des identifiants détermine \texttt{trade\_value}.
    \item Il n'existe pas de dépendance fonctionnelle violant cette condition.
    \item \textbf{Conclusion} : le schéma respecte la forme normale de \textbf{Boyce-Codd (BCNF)}.
  \end{itemize}
\end{frame}

\section{Script de création et peuplement SQL}
\begin{frame}[fragile]
  \frametitle{Création de la table \texttt{Country}}
\begin{verbatim}
CREATE TABLE Country(
   id_Country INTEGER PRIMARY KEY,
   Country VARCHAR(50),
   ISO2 CHAR(2),
   ISO3 CHAR(3)
);
\end{verbatim}
\end{frame}

\begin{frame}[fragile]
  \frametitle{Création de la table \texttt{Indicator}}
\begin{verbatim}
CREATE TABLE Indicator(
   id_Indicator INTEGER PRIMARY KEY,
   Indicator_ VARCHAR(80),
   Source VARCHAR,
   Units VARCHAR(50)
);
\end{verbatim}
\end{frame}

\begin{frame}[fragile]
  \frametitle{Création de la table \texttt{CTS}}
\begin{verbatim}
CREATE TABLE CTS(
   id_CTS INTEGER PRIMARY KEY,
   CTS_Code VARCHAR(6),
   CTS_Name VARCHAR(100),
   CTS_Full_Descriptor VARCHAR(150)
);
\end{verbatim}
\end{frame}

\begin{frame}[fragile]
  \frametitle{Création de la table \texttt{Trade\_Flow}}
\begin{verbatim}
CREATE TABLE Trade_Flow(
   id_Trade INTEGER PRIMARY KEY,
   Trade_Flow VARCHAR(20),
   Scale VARCHAR(5)
);
\end{verbatim}
\end{frame}

\begin{frame}[fragile]
  \frametitle{Création de la table \texttt{Year}}
\begin{verbatim}
CREATE TABLE Year(
   id_Year INTEGER PRIMARY KEY,
   Year DATE
);
\end{verbatim}
\end{frame}

\begin{frame}[fragile]
  \frametitle{Création de la table \texttt{Echanger\_Avec}}
\begin{verbatim}
CREATE TABLE Echanger_Avec (
    id_Country_1 INTEGER REFERENCES Country(id_Country),
    id_Country_2 INTEGER REFERENCES Country(id_Country),
    Pays_Origine INTEGER REFERENCES Country(id_Country),
    id_Indicator INTEGER REFERENCES Indicator(id_Indicator),
    id_CTS INTEGER REFERENCES CTS(id_CTS),
    id_Trade INTEGER REFERENCES Trade_Flow(id_Trade),
    id_Year INTEGER REFERENCES Year(id_Year),
    trade_value DOUBLE PRECISION,
    PRIMARY KEY(id_Country_1, id_Country_2, Pays_Origine,
                id_Indicator, id_CTS, id_Trade, id_Year),
    CHECK (id_Country_1 < id_Country_2)
);
\end{verbatim}
\end{frame}

\begin{frame}[fragile]
  \frametitle{Création de la table \texttt{Donnee\_Pays}}
\begin{verbatim}
CREATE TABLE Donnee_Pays (
    id_Country INTEGER REFERENCES Country(id_Country),
    id_Indicator INTEGER REFERENCES Indicator(id_Indicator),
    id_CTS INTEGER REFERENCES CTS(id_CTS),
    id_Trade INTEGER REFERENCES Trade_Flow(id_Trade),
    id_Year INTEGER REFERENCES Year(id_Year),
    trade_value DOUBLE PRECISION,
    PRIMARY KEY(id_Country, id_Indicator, id_CTS, 
    id_Trade, id_Year)
);
\end{verbatim}
\end{frame}


\begin{frame}
  \frametitle{Peuplement de la base}
  \begin{itemize}
    \item Utilisation de \texttt{COPY}, \texttt{INSERT INTO} avec jointures.
    \item Alignement avec les traitements Python (\texttt{melt + filtering}).
    \item Cas particuliers traités : 
      \begin{itemize}
        \item Échanges d’un pays avec lui-même
        \item Lignes \texttt{NULL} / Trade Flow = Not Applicable
      \end{itemize}
    \item Difficultés rencontrées : [à compléter]
  \end{itemize}
\end{frame}

\section{Interrogation et visualisation des données}
\begin{frame}
  \frametitle{Requêtes d’analyse}
  \begin{itemize}
    \item Nombre d’échanges par année
    \item Top pays exportateurs/importateurs
    \item Évolution des indicateurs par catégorie CTS
    \item Comparaison SQL x CSV avec Python
  \end{itemize}
\end{frame}

\begin{frame}
  \frametitle{Visualisation}
  \begin{itemize}
    \item Utilisation de Metabase pour interroger la BD
    \item Utilisation de Tableau pour valider les valeurs issues du CSV
    \item Comparaison SQL / Python → vérification de l’intégrité
    \item Exemples visuels : [insérer graphiques ou screenshots]
  \end{itemize}
\end{frame}

\section{Conclusion}
\begin{frame}
  \frametitle{Bilan du projet}
  \begin{itemize}
    \item Base de données fonctionnelle et fidèle au CSV (dans nos rêves)
    \item Respect des étapes du processus de modélisation
    \item Difficultés sur l’import (pivot, valeurs nulles, sens du flux)
    \item Vérification croisée avec Python
    \item Améliorations possibles : [à compléter]
  \end{itemize}
\end{frame}

\begin{frame}
  \frametitle{Merci pour votre attention}
  \begin{center}
    \huge \textbf{Questions ?}
  \end{center}
\end{frame}

\end{document}